\subsection{Slashing Mechanism}

The slashing mechanism is used to enforce the Network's security and stability.
It is applied to both Normal Nodes and Public Good Nodes, albeit in slightly different ways.

\subsubsection{Demotion}
A demotion is automatically triggered when a Node fails to meet the requirements set by the Network.
This can be due to a variety of reasons, including but not limited to: 1) the Node is offline for an extended period of time; 2) the Node is not serving requests in a timely manner; 3) the Node is serving requests but with incorrect information.

The Node's \reliabilityScore\ will be negatively impacted, diminishing its likelihood of receiving requests on the \gls{DSL}.
This, in turn, will reduce the Node's potential \operationReward\ and \stakingReward\ allocated from the \gls{VSL}.


\subsubsection{Slashing}

A slashing is automatically triggered when a Node is repeatedly demoted.
Should a slashing occur, the Node's \operationPool\ and \stakingPool\ will be slashed by percentages determined by the Network. Public Good Nodes are not subject to token slashing.

The Node's \reliabilityScore\ will be set to 0, effectively preventing it from receiving requests on the \gls{DSL} during the current epoch.

The disposition of the slashed tokens is as follows:
\begin{itemize}
    \item a portion of the slashed tokens will be burned, with the amount determined by the Network
    \item a portion of the slashed tokens will go to the reporter, provided the Node’s misconduct was not auto-detected by the Network
    \item the remaining portion of the slashed tokens will go to the \publicGoodPool
\end{itemize}

\subsubsection{Challenge}
When a Node is slashed, its \glsfmtlong{NO} has the ability to challenge the slashing within a certain period. A successful challenge will result in the slashed tokens being returned to the Node's \operationPool\ and \stakingPool.