
\subsection{Staking, Trust, and Chip}
Network participants are incentivized to secure and improve the Network with their \$RSS3 tokens.
The staking contract \cite{stakingcontract} is responsible for managing the staking, trust, and Chip issurance.

\subsubsection{Staking}

A Normal Node accepts staking into its \stakingPool;
The amount of staked \$RSS3 signifies its quality and reliability, and this increases the likelihood of receiving requests for a Node.

\subsubsection{Trust}

A Public Good Node does not have a \glsfmtlong{SP} and does not participate in any form of incentivization. Instead, network participants may choose to entrust such a Node, and their tokens are stored in a \glsfmtlong{PGP}. The trust level affects the likelihood of routing requests to a Public Good Node.

\subsubsection{Chip}

A Chip \chip\ is an ERC-721 Non-Fungible Token (NFT) representing a network participant's stake in a particular Node.
Refer to its contract \cite{chipscontract}.

\paragraph{Minting}
When a network participant stakes or entrusts tokens to a Node \node, the participant automatically receives a Chip-\node\ ($\chip_\node$).
Its value is dynamically calculated by its weight in $\stakingPool$:

\begin{equation}
    \label{eq:chip_weight}
    \chip_{\node_\text{weight}} =
    \frac{\staking}{\stakingPool + \staking}
\end{equation}

Where $\staking \in \mathbb Z_{> 0}$ is always a non-negative integer.

\paragraph{Redemption}
A Chip can be redeemed for its underlying staked or entrusted tokens at any time, subject to a waiting period imposed by the Network.
The redemption amount may be different from the original staking or entrusting amount due to the change of the underlying \stakingPool\ balance:

\begin{equation}
    \label{eq:chip_redeem}
    \staking_\text{redeem} = \chip_{\node_\text{weight}} \times \stakingPool
\end{equation}

% {
% \renewcommand{\arraystretch}{1.5}
% \begin{table*}[h]
%     \resizebox{\textwidth}{!}{
%         \begin{tabulary}{\textwidth}{|p{6cm}|p{5cm}|p{5cm}|}
%             \hline
%             & \textbf{Node in Normal Mode} & \textbf{Node in Public Good mode} \\ \hline

%             Who can operate? &
%             Anyone &
%             Anyone \\ \hline

%             Is a deposit required for operating a Node? &
%             Yes &
%             No \\ \hline

%             Is the deposit considered as staking, making it eligible for rewards from its own \stakingPool? &
%             No &
%             N/A \\ \hline

%             Will the Node be slashed? &
%             Yes, its deposit and \stakingPool will be slashed. A Node may be demoted to receive fewer requests.
%             & No, but a Node may be demoted to receive fewer requests. \\ \hline

%             Does the Node accept staking? &
%             Yes. The staked tokens go to the Node’s \stakingPool. RSS3-X (X being the Node’s name) Chips are issued to the stakers after staking. &
%             No, as such a Node does not have a \stakingPool. Instead, stakers stake to the \publicGoodPool. RSS3-Public Good Chips are issued to the stakers after staking. \\ \hline

%             Can the \glsfmtlong{NO} set a tax \taxRate? &
%             Yes &
%             No, a universal tax is determined by the Network. \\ \hline

%             Does it have an \glsfmtlong{OP} \operationPool? &
%             Yes &
%             No, its \glsfmtlong{OR} go to [X] \\ \hline

%             Does it have a \glsfmtlong{SP} \stakingPool? &
%             Yes &
%             No, but a \glsfmtlong{PGP} with a universal incentive rate. \\ \hline
%         \end{tabulary}
%     }
%     \caption{Comparison of two Node Operation modes.}
%     \label{table:node_modes}
% \end{table*}
% }
