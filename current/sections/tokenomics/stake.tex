
\subsection{Staking, Trust, and Chip}
Network participants are incentivized to secure and improve the Network with their \$RSS3 tokens.

\subsubsection{Staking}

A Normal Node accepts staking into its \stakingPool;
The amount of staked \$RSS3 signifies its quality and reliability, and this increases the likelihood of receiving requests for a Node.

\subsubsection{Trust}

A Public Good Node does not have a \glsfmtlong{SP} and does not participate in any form of incentivization. Instead, network participants may choose to entrust such a Node, and their tokens are stored in a \glsfmtlong{PGP}. The trust level affects the likelihood of routing requests to a Public Good Node.

\subsubsection{Chip}

A Chip \chip\ is an ERC-721 Non-Fungible Tokens (NFTs) representing a network participant's stake in a particular Node.
When a network participant stakes or entrusts tokens to a Node \node, the participant automatically receives Chips RSS3-\node\ ($\chip_\node$).

\paragraph{Minting}
A single Chip is minted for every staking transaction, irrespective of the quantity staked. The value of each Chip is determined by the ratio \radio\ of the current total amount of stake \$RSS3 tokens to the Node current value of the Chip:

\begin{equation}
    \label{eq:chip_radio}
    \chip_{\radio_{\text{mint}}} = 
    \frac{\staking_{\text{mint}}}{\chip_{\node_\text{value}}}
\end{equation}

\paragraph{Redemption}
A Chip can be redeemed for its underlying staked or entrusted tokens at any time, subject to a waiting period imposed by the Network.
The redemption amount may be different from the original staking or entrusting amount due to the change of the Node current value of the Chip, which follows the same formula as \Cref{eq:chip_radio}:

\begin{equation}
    \label{eq:chip_redeem}
    \staking_{\text{redeem}} = 
    \chip_{\node_\text{value}} * \chip_{\radio_\text{mint}}
\end{equation}

% {
% \renewcommand{\arraystretch}{1.5}
% \begin{table*}[h]
%     \resizebox{\textwidth}{!}{
%         \begin{tabulary}{\textwidth}{|p{6cm}|p{5cm}|p{5cm}|}
%             \hline
%             & \textbf{Node in Normal Mode} & \textbf{Node in Public Good mode} \\ \hline

%             Who can operate? &
%             Anyone &
%             Anyone \\ \hline

%             Is a deposit required for operating a Node? &
%             Yes &
%             No \\ \hline

%             Is the deposit considered as staking, making it eligible for rewards from its own \stakingPool? &
%             No &
%             N/A \\ \hline

%             Will the Node be slashed? &
%             Yes, its deposit and \stakingPool will be slashed. A Node may be demoted to receive fewer requests.
%             & No, but a Node may be demoted to receive fewer requests. \\ \hline

%             Does the Node accept staking? &
%             Yes. The staked tokens go to the Node’s \stakingPool. RSS3-X (X being the Node’s name) Chips are issued to the stakers after staking. &
%             No, as such a Node does not have a \stakingPool. Instead, stakers stake to the \publicGoodPool. RSS3-Public Good Chips are issued to the stakers after staking. \\ \hline

%             Can the \glsfmtlong{NO} set a tax \taxRate? &
%             Yes &
%             No, a universal tax is determined by the Network. \\ \hline

%             Does it have an \glsfmtlong{OP} \operationPool? &
%             Yes &
%             No, its \glsfmtlong{OR} go to [X] \\ \hline

%             Does it have a \glsfmtlong{SP} \stakingPool? &
%             Yes &
%             No, but a \glsfmtlong{PGP} with a universal incentive rate. \\ \hline
%         \end{tabulary}
%     }
%     \caption{Comparison of two Node Operation modes.}
%     \label{table:node_modes}
% \end{table*}
% }
