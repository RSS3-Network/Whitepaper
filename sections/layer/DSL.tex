\section{\glsfmtlong{DSL}}
\label{sec:DSL}

The \gls{DSL} is responsible for Open Information life cycle management, which includes indexing, transformation, storage, dissemination, and consumption \cite{nationalinstituteofstandardsandtechnology2016Information}. The \gls{DSL} is formed by two components (see \cref{subsec:SN} and \cref{subsec:GI}), and uses the \gls{UMS} (see \cref{subsec:UMS}) to structure the information for applications in social, search, AI and beyond.

\subsection{\glsfmtfull{SN}}
\label{subsec:SN}

An \gls{SN} is responsible for indexing, transforming, storing, and ultimately serving the Open Information to the end users. Each \gls{SN} operates a number of workers that index and structure information from \gls{PDS}, stores the information, and provides interfaces for access. Workers are community-maintained ``rules'' that define how information is indexed and transformed into the \gls{UMS} format. 

Since each \gls{SN} is independent, it is possible for different \glspl{SN} to operate different workers that cover different \glspl{PDS}, their coverage include, but not limited to, decentralized networks, federated networks, and centralized networks (with permissionless access).

This design enables node operation to be flexible, accessible and affordable, in turn, offering a high degree of decentralization and robustness.

\subsection{\glsfmtfull{GI}}
\label{subsec:GI}

A \gls{GI} is responsible for facilitating coordination among \glspl{SN} and engaging with the \gls{VSL}, and performs the following functions:
\begin{enumerate}
    \item A load balancer and query router for end users to retrieve information from \glspl{SN}.
    \item A supervisor for \glspl{SN} to ensure the quality of service.
    \item A settler for submitting work and slash records to the \gls{VSL}.
\end{enumerate}

The unique network architecture of the \gls{DSL} demands \glspl{GI} to be equipped with more computational capabilities, in order to work out the optimal route for end users to retrieve information from \glspl{SN}.

\subsection{\glsfmtfull{UMS}}
\label{subsec:UMS}

Open Information, indexed from multiple \glspl{PDS}, is structured by \glspl{SN} into the \gls{UMS} format for interoperability.

\glspl{PDS} use different data structures, within a \gls{PDS}, there might be multiple products, services and protocols that leverage a different data structure to suit their needs. This means limited interoperability, and developers need to look into each and every data structure, when it comes to building. This lack of standardization means developers must investigate each unique structure individually when building applications, which is not scalable.

The \gls{UMS} addresses this issue by offering a unified set of data structures that serve as an abstraction. This abstraction simplifies the integration process, making it more manageable and scalable for developers to work with data across various data sources.

For the complete set of the \gls{UMS}, refer to \url{https://docs.rss3.io/docs/unified-metadata-schemas}.
