\section{\glsfmtlong{VSL}}
\label{sec:VSL}

The \glsfirst{VSL}, commonly referred to as the RSS3 Chain, is an Ethereum Layer 2 blockchain built with OP Stack uisng Celestia as the data availability layer.
It is responsible for handling value derived from \glsfmtlong{OI} activities and applications, establishing a healthy ownership economy for the Network.

In this section, we focus on the intentions behind the \gls{VSL}'s incentive mechanism, which is designed to promote stable Node Operations to maintain the Network, and to encourage network participants to secure the Network via staking \$RSS3.
We introduce the detailed tokenomics separately in \Cref{sec:tokenomics}.

The \glsfmtlong{R3N} allocates a portion of \$RSS3 total supply to incentivize network participants, referred to as the \glsfirst{NR},
are allocated into reward pools: the \glsfirst{OP} and the \glsfirst{SP} for Normal Nodes, or the \glsfirst{PGP} for Public Good Nodes.
See \Cref{fig:network-rewards} for an illustration and \Cref{subsec:reward_pools} for details on Reward Pools.
The calculation of \glsfmtlong{NR} are described in \Cref{subsec:network_rewards}.

\subsection{Node Operation}
\Glsfmtlong{NO}s are incentivized to operate and maintain the Network by receiving \$RSS3 as rewards.
\begin{enumerate}
    \item Anyone can become a \glsfmtlong{NO} to launch an RSS3 Node and join the RSS3 Network without requiring prior permission.
    \item A \glsfmtlong{NO} has the ability to configure Node's coverage, which directly influences the Node's capability to respond to various types of requests. A broader coverage means more computational resources are required, and a higher chance of receiving requests.
    \item A Node can be operated in either a Normal mode or a Public Good mode. A Normal Node is eligible for \glsfmtlong{NR}, but requires a deposit of \$RSS3. A Public Good Node is ineligible for \glsfmtlong{NR}, but requires no deposit.
    \item A Normal Node has a corresponding \operationPool\ and a \stakingPool. All Public Good Nodes collectively share a single \publicGoodPool.
\end{enumerate}

\subsection{Node Staking}
Network participants are incentivized to stake \$RSS3 to secure the Network by receiving \$RSS3 as rewards.
\begin{enumerate}
    \item A Normal Node accepts staking into its \stakingPool, the amount of staked \$RSS3 signifies its quality. Higher quality Nodes handle more requests.
    \item A Public Good Node does not have Reward Pools and does not participate in any form of incentivization. Staking into a \glsfmtlong{PGP} is accepted, and the stakers can assign their trust to any Public Good Node. Higher trust Nodes handle more requests.
\end{enumerate}

{
\renewcommand{\arraystretch}{1.5}
\begin{table*}[h]
    \resizebox{\textwidth}{!}{
        \begin{tabulary}{\textwidth}{|p{6cm}|p{5cm}|p{5cm}|}
            \hline
            & \textbf{Node in Normal Mode} & \textbf{Node in Public Good mode} \\ \hline

            Who can operate? &
            Anyone &
            Anyone \\ \hline

            Is a deposit required for operating a Node? &
            Yes &
            No \\ \hline

            Is the deposit considered as staking, making it eligible for rewards from its own \stakingPool? &
            No &
            N/A \\ \hline

            Will the Node be slashed? &
            Yes, its deposit and \stakingPool will be slashed. A Node may be demoted to receive fewer requests.
            & No, but a Node may be demoted to receive fewer requests. \\ \hline

            Does the Node accept staking? &
            Yes. The staked tokens go to the Node’s \stakingPool. RSS3-X (X being the Node’s name) Chips are issued to the stakers after staking. &
            No, as such a Node does not have a \stakingPool. Instead, stakers stake to the \publicGoodPool. RSS3-Public Good Chips are issued to the stakers after staking. \\ \hline

            Can the \glsfmtlong{NO} set a tax \tax? &
            Yes &
            No, a universal tax is determined by the Network. \\ \hline

            Does it have an \glsfmtlong{OP} \operationPool? &
            Yes &
            No, its \glsfmtlong{OR} go to [X] \\ \hline

            Does it have a \glsfmtlong{SP} \stakingPool? &
            Yes &
            No, but a \glsfmtlong{PGP} with a universal incentive rate. \\ \hline
        \end{tabulary}
    }
    \caption{Comparison of two Node Operation modes.}
    \label{table:node_modes}
\end{table*}
}
