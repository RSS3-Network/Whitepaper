\section{\glsfmtlong{VSL}}
\label{sec:VSL}

The \glsfirst{VSL}, commonly referred to as the RSS3 Chain, is an Ethereum Layer 2 blockchain built with OP Stack uisng Celestia as the data availability layer.
It is responsible for handling value derived from \glsfmtlong{OI} activities and applications, establishing a healthy ownership economy for the Network.

In this section, we focus on the intentions behind the \gls{VSL}'s incentive mechanism, which is designed to promote stable Node Operations to maintain the Network, and to encourage network participants to secure the Network via staking \$RSS3.
We introduce the detailed tokenomics separately in \Cref{sec:tokenomics}.

The \glsfmtlong{R3N} allocates a portion of \$RSS3 total supply to incentivize network participants, referred to as the \glsfirst{NR}, 
are allocated into reward pools: the \glsfirst{OP} and the \glsfirst{SP} for Normal Nodes, or the \glsfirst{PGP} for Public Good Nodes.
See \Cref{fig:network-rewards} for an illustration and \Cref{subsec:reward_pools} for details on Reward Pools.
The calculation of \glsfmtlong{NR} are described in \Cref{sec:tokenomics}.

\subsection{Node Operation}
\Glsfmtlong{NO}s are incentivized to operate and maintain the Network by receiving \$RSS3 as rewards.
\begin{enumerate}
    \item Anyone can become a \glsfmtlong{NO} to launch an RSS3 Node and join the RSS3 Network without requiring prior permission.
    \item A \glsfmtlong{NO} has the ability to configure Node's coverage, which directly influences the Node's capability to respond to various types of requests. A broader coverage means more computational resources are required, and a higher chance of receiving requests.
    \item A Node can be operated in either a Normal mode or a Public Good mode. A Normal Node is eligible for \glsfmtlong{NR}, but requires a deposit of \$RSS3. A Public Good Node is ineligible for \glsfmtlong{NR}, but requires no deposit.
    \item A Normal Node has a corresponding \operationPool\ and a \stakingPool. All Public Good Nodes collectively share a single \publicGoodPool.
\end{enumerate}

\subsection{Node Staking}
Network participants are incentivized to stake \$RSS3 to secure the Network by receiving \$RSS3 as rewards.
\begin{enumerate}
    \item A Normal Node accepts staking into its Reward Pool, the amount of staked \$RSS3 signifies its quality. Higher quality Nodes handle more requests.
    \item A Public Good Node does not have a Reward Pool and does not participate in any form of incentivization. Staking into a \glsfmtlong{PGP} is accepted, and the stakers can assign their trust to any Public Good Node. Higher trust Nodes handle more requests.
\end{enumerate}

{
\renewcommand{\arraystretch}{1.5}
\begin{table*}[h]
    \resizebox{\textwidth}{!}{
        \begin{tabulary}{\textwidth}{|p{6cm}|p{5cm}|p{5cm}|}
            \hline
            & \textbf{Node in Normal Mode} & \textbf{Node in Public Good mode} \\ \hline
            Who can operate? & Anyone & Anyone \\ \hline
            Can \glsfmtlong{NO} specify the coverage? & Yes & Yes \\ \hline
            Is a deposit required? & Yes & No \\ \hline
            Is the deposit considered as staking, making it eligible for rewards from its own \stakingPool? & No & N/A \\ \hline
            Will the Node be slashed? & Yes, its deposit and \stakingPool will be slashed. A Node may be demoted to receive fewer requests. & No, but a Node may be demoted to receive fewer requests. \\ \hline
            Does the Node accept staking? & Yes. The staked tokens go to the Node’s \stakingPool. RSS3-X (X being the Node’s name) Chips are issued to the stakers after staking. & No, as such a Node does not have a \glsfmtlong{SP}. Instead, stakers stake to a \glsfmtlong{PGP}. RSS3-Public Good Chips are issued to the stakers after staking. \\ \hline
            Can \glsfmtlong{NO} set an \glsfirst{Tax}? & Yes & No, a universal tax is determined by the Network. \\ \hline
            Does it have an \glsfmtlong{OP}? & Yes & No, operator rewards go to [X] \\ \hline
            Does it have a \glsfmtlong{SP}? & Yes & No, but a \glsfmtlong{PGP} with a universal incentive rate. \\ \hline
        \end{tabulary}
    }
    \caption{Comparison of two Node operation modes.}
    \label{table:node_modes}
\end{table*}
}

\subsection{Reward Pools}
\label{subsec:reward_pools}

This section introduces the three reward pools: the \glsfirst{OP}, the \glsfirst{SP}, and the \glsfirst{PGP}. See \Cref{fig:network-rewards} for an illustration.

{
\begin{figure}[tb!]
    \centering
    \includegraphics[width=0.9\columnwidth]{figures/network-rewards.png}
    \caption{RSS3 \glsfmtlong{NR} distribution.
    The \glsfmtlong{NR} are allocated into two reward pools: the \glsfirst{OP} and the \glsfirst{SP} for Normal Nodes, or the \glsfirst{PGP} for Public Good Nodes.
    See \Cref{subsec:reward_pools} for details.}
    \label{fig:network-rewards}
\end{figure}
}

\subsubsection{\glsfmtlong{OP} (\operationPool)}
\label{subsubsec:operation_pool}

An \glsfirst{OP} is used to store tokens that are allocated to a Normal Node from three sources: 1) the \gls{Fee} collected from requests served on the \gls{DSL}, denoted \work; 3) the \glsfmtlong{NR} allocated based on the Node’s work; 3) the \gls{Tax} collected from the Node's \stakingPool.

The allocation of \glsfmtlong{NR} into a Node's \operationPool\ at the end of each epoch, is determined by the \glsfmtlong{N} (\work), in proportion to the total number of requests served on the \gls{DSL}.

The \glsfmtlong{NO} can set a tax rate, \tax, which is applied to its \stakingPool.
The tax applies to the \glsfmtlong{NR} allocated to the Node's \stakingPool, not the staked tokens (REP-1: Chip Redemption Tax).

Only the corresponding \glsfmtlong{NO} can withdraw tokens from its \operationPool, and the withdrawal is subject to a waiting period imposed by the Network.

\subsubsection{\glsfmtlong{SP} (\stakingPool)}
\label{subsubsec:staking_pool}

A \glsfirst{SP} is used to store staked tokens for a Normal Node. Network participants can stake tokens into a Normal Node's \stakingPool\ to increase the Node's chance to receive requests on the \gls{DSL}.

The allocation of \glsfmtlong{NR} into a Node's \stakingPool\ at the end of each epoch, is determined by the size of the Node's \stakingPool, in proportion to the total staked tokens on the \gls{VSL}.
A tax is then applied to the received Rewards, with the rate set by its \glsfmtlong{NO}.

\subsubsection{\glsfmtlong{PGP} (\publicGoodPool)}

A \glsfirst{PGP} is a unique reward pool that is shared by all Public Good Nodes.

As Public Good Nodes do not have their own \stakingPool, network participants stake into the \publicGoodPool\ to assign their trust to any Public Good Node.
