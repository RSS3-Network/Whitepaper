\subsection{Slashing}

Slashing is a mechanism used to enforce the Network's security and stability. It is applied to both Normal Nodes and Public Good Nodes, albeit in slightly different ways.

A slashing occurs when a Node fails to meet the requirements set by the Network. This can be due to a variety of reasons, including but not limited to: 1) the Node is offline for an extended period of time; 2) the Node is not serving requests in a timely manner; 3) the Node is serving requests but with incorrect information.

Should a slashing occur, the Node's \deposit\%, \stakingPool\%, and \networkReward\ will be slashed by precentages determined by the Network.
The Node's \reliabilityScore\ will be negatively impacted, diminishing its likelihood of receiving requests on the \gls{DSL}.

The disposition of the slashed tokens is as follows:
\begin{itemize}
    \item a portion of the slashed tokens will be burned, the amount is determined by the Network
    \item a portion of the slashed tokens will go to the reporter, provided the Node’s misconduct was not auto-detected by the Network
    \item the remaining portion of the slashed tokens will go to the \publicGoodPool
\end{itemize}

% TODO: add slashing into formulas7