\section{Tokenomics}
\label{sec:tokenomics}

In this seciton, we introduce the detailed tokenomics of the \glsfmtlong{R3N}. We present the concept of Reward Pools, the \glsfmtlong{NR}'s calculation and distribution formulas, and the slashing mechanism employed to enforce network security and stability.

\subsection{Reward Pools}
\label{subsec:reward_pools}

This section introduces the three reward pools: the \glsfirst{OP}, the \glsfirst{SP}, and the \glsfirst{PGP}. See \Cref{fig:network-rewards} for an illustration.

    {
        \begin{figure}[tb!]
            \centering
            \includegraphics[width=0.9\columnwidth]{figures/network-rewards.png}
            \caption{RSS3 \glsfmtlong{NR} distribution.
                The \glsfmtlong{NR} are allocated into two reward pools: the \glsfirst{OP} and the \glsfirst{SP} for Normal Nodes, or the \glsfirst{PGP} for Public Good Nodes.
                See \Cref{subsec:reward_pools} for details.}
            \label{fig:network-rewards}
        \end{figure}
    }

\subsubsection{\glsfmtlong{OP} (\operationPool)}
\label{subsubsec:operation_pool}

An \glsfirst{OP} is used to store tokens that are allocated to a Normal Node from three sources: 1) the \gls{Fee} collected from requests served on the \gls{DSL}, denoted \work; 3) the \glsfmtlong{NR} allocated based on the Node’s work; 3) the \gls{Tax} collected from the Node's \stakingPool.

The allocation of \glsfmtlong{NR} into a Node's \operationPool\ at the end of each epoch, is determined by the \glsfmtlong{N} (\work), in proportion to the total number of requests served on the \gls{DSL}.

The \glsfmtlong{NO} can set a tax rate, \taxRate, which is applied to its \stakingPool.
The tax applies to the \glsfmtlong{NR} allocated to the Node's \stakingPool, not the staked tokens (REP-1: Chip Redemption Tax).

Only the corresponding \glsfmtlong{NO} can withdraw tokens from its \operationPool, and the withdrawal is subject to a waiting period imposed by the Network.

\subsubsection{\glsfmtlong{SP} (\stakingPool)}
\label{subsubsec:staking_pool}

A \glsfirst{SP} is used to store staked tokens for a Normal Node. Network participants can stake tokens into a Normal Node's \stakingPool\ to increase the Node's chance to receive requests on the \gls{DSL}.

The allocation of \glsfmtlong{NR} into a Node's \stakingPool\ at the end of each epoch, is determined by the size of the Node's \stakingPool, in proportion to the total staked tokens on the \gls{VSL}.
A tax is then applied to the received Rewards, with the rate set by its \glsfmtlong{NO}.

\subsubsection{\glsfmtlong{PGP} (\publicGoodPool)}

A \glsfirst{PGP} is a unique reward pool that is shared by all Public Good Nodes.

As Public Good Nodes do not have their own \stakingPool, network participants entrust their tokens to the \publicGoodPool\ and signify their support to a designated Public Good Node.

\subsection{\glsfmtfull{NR}}
\label{subsec:network_rewards}

In \Cref{sec:VSL}, we describe the intentions behind the \gls{VSL}'s incentive mechanism, here we introduce the detailed \glsfmtlong{NR} calculation and distribution formulas separately.

The \glsfmtlong{NR} \networkReward\ consists of three parts:
\begin{equation}
    \label{eq:network_rewards}
    \networkReward = (\operationReward + \stakingReward) + \trustingReward
\end{equation}

See \Cref{fig:network-rewards} for an illustration. The allocation to each part is determined by the Network, and is subject to potential future changes.

\subsubsection{\glsfmtfull{OR}}
To encourage Normal Nodes to operate and maintain the Network, \operationReward\ is allocated to a Node's \operationPool\ in proportion to its \glsfirst{Fee} collected on the \gls{DSL} during the last \epoch.

\begin{equation}
    \label{eq:operation_weight}
    \tilde{\fee}_\nodeAtEpoch =
    \log_{2}
    (
    \frac{\fee_\nodeAtEpoch}
    {\sum_{x=0}^{\infty} \fee_{x, \epoch}} + 1
    ) * G
\end{equation}

$\tilde{\fee}_\nodeAtEpoch$ denotes the normalized work contribution for a given Normal Node \node, at the end of a given epoch $\epoch$. $G$ is a constant equal to $\ln(2) \approx 0.693147$ used to offset the effect of replacing $\ln$ with $\log_2$, as the former is more costly in terms of gas when it comes to on-chain computation.

\begin{equation}
    \label{eq:operation_rewards}
    \networkReward_{\operation|\node, \epoch} =
    \frac{\tilde{\fee}_{\node, \epoch}}
    {\sum_{x=0}^{\infty} \tilde{\fee}_{x, \epoch}}
    * \networkReward_{\operation, \epoch}
\end{equation}

$\networkReward_{\operation|\node, \epoch}$ therefore denotes the Operation Rewards for a given Normal Node \node, at the end of a given epoch $\epoch$.

\subsubsection{\glsfmtfull{SR}}

To encourage participation from all network participants to increase the Network's reliability, \stakingReward\ is allocated to a Normal Node's \stakingPool\ in proportion to the amount of staked tokens in the entire Network during the last \epoch.

\begin{equation}
    \label{eq:staking_rewards}
    \networkReward_{\staking|\node, \epoch} =
    \frac{\pool_{\staking|\node, \epoch}}
    {\sum_{x=0}^{\infty} \pool_{\staking|x, \epoch}}
    * \networkReward_{\staking, \epoch}
\end{equation}

$\networkReward_{\staking|\node, \epoch} * (1 - \taxRate_{\node, \epoch})$ therefore denotes the Staking Rewards for a given Normal Node \node, at the end of a given epoch $\epoch$.

\subsubsection{\glsfmtfull{TR}}

To encourage participation from all network participants to increase the Network's reliability and support Public Goods provision, \trustingReward\ is allocated to the \publicGoodPool\ in proportion to the amount of entrusted tokens in the entire Network during the last \epoch.

\begin{equation}
    \label{eq:trusting_rewards}
    \networkReward_{\trusting|\node, \epoch} =
    \frac{\pool_{\trusting|\node, \epoch}}
    {\sum_{x=0}^{\infty} \pool_{\trusting|x, \epoch}}
    * \networkReward_{\trusting, \epoch}
\end{equation}

$\networkReward_{\trusting|\node, \epoch}$ therefore denotes the Trusting Rewards for a given Public Good Node \node, at the end of a given epoch $\epoch$.

\subsubsection{Taxation (\tax)}

The tax rate \taxRate\ is set by the \glsfmtlong{NO} of a Normal Node, and is applied to the \glsfmtlong{SR} allocated to its \stakingPool.
The amount of tax collectible is capped at a maximum of \taxCap\ times the amount of the current deposit. \taxCap\ is set by the Network.

\begin{equation}
    \tax_{\node, \epoch} =
    \min(\deposit_{\node, \epoch} * \taxCap_\epoch , \networkReward_{\staking|\node, \epoch} * \taxRate_{\node, \epoch})
\end{equation}

The total amount of tokens allocated to a Normal Node's \operationPool\ for a given epoch $\epoch$ is therefore:

\begin{equation}
    \networkReward_{total|\node, \epoch} =
    \networkReward_{\operation|\node, \epoch}
    + \fee_{\node, \epoch}
    + \tax_{\node, \epoch}
\end{equation}

\subsection{Chip}

\subsection{Slashing}

Slashing is a mechanism used to enforce network security and stability. It is applied to both Normal Nodes and Public Good Nodes, albeit in slightly different ways.

A slashing occurs when a Node fails to meet the requirements set by the Network. This can be due to a variety of reasons, including but not limited to: 1) the Node is offline for an extended period of time; 2) the Node is not serving requests in a timely manner; 3) the Node is serving requests but with incorrect information.

Should a slashing occur, the Node's \deposit\, \stakingPool\, and \networkReward\ will be slashed by precentages determined by the Network.
The Node's \reliabilityScore\ will be negatively impacted, diminishing its likelihood of receiving requests on the \gls{DSL}.

The disposition of the slashed tokens is as follows:
\begin{itemize}
    \item a portion of the slashed tokens will be burned, the amount is determined by the Network
    \item a portion of the slashed tokens will go to the reporter, provided the Node’s misconduct was not auto-detected by the Network
    \item the remaining portion of the slashed tokens will go to the \publicGoodPool
\end{itemize}