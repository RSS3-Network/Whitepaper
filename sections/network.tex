\section{\glsfmtlong{R3N}}

The \glsfmtlong{R3N} is a decentralized network that is formed by two Sublayers: the \glsfirst{DSL} and the \glsfirst{VSL}.
This novel network structure is the product of a series of research and development experiments, that were conducted to address the challenges faced by the Open Web.

\gls{OI} is typically found across various types of networks, including decentralized, federated, and centralized networks that allow permissionless access.
The \gls{DSL} is responsible for indexing and structuring \gls{OI} for interoperability.
This is achieved by introducting a crucial standard, known as the \gls{UMS}, see \Cref{subsec:UMS}, enabling network-agnostic applications to be built on top of the \gls{DSL}.
The \gls{DSL} then leverages the \gls{VSL}, see \Cref{sec:VSL}, to build an ownership economy for the \gls{OW}.

\subsection{\$RSS3}
\$RSS3 is the Network's native utility token. It is used to cover gas, pay request fees, operate nodes, participate in staking and trust, distribute incentives, and engage in various network activities. See \Cref{sec:tokenomics} for more details.

\subsection{\Glsfmtfull{Epoch}}

An \glsfirst{Epoch} is a period of time used as a reference to measure the RSS3 Network’s operation, during which the Network's parameters are fixed.
The duration of an epoch is determined by the Network, and is subject to potential future changes.

At the end of each \epoch, the Network will distribute the \gls{NR} to the \glsfmtlong{R3N}'s participants, and update the Network's parameters when necessary.