
% The preceding line is only needed to identify funding in the first footnote. If that is unneeded, please comment it out.
\renewcommand\IEEEkeywordsname{Keywords}

\usepackage[noadjust]{cite}
\renewcommand{\citepunct}{,\penalty\citepunctpenalty\,}
\renewcommand{\citedash}{--}
\usepackage{caption}
\usepackage{url}
\usepackage{titletoc}
\usepackage{amsmath,amssymb,amsfonts}
\usepackage{algorithmic}
\usepackage{graphicx}
\usepackage{textcomp}
\usepackage{bm}
\usepackage{booktabs}
\usepackage{multirow}
\usepackage{tikz}
\usepackage{pgfplots}
\def\BibTeX{{\rm B\kern-.05em{\sc i\kern-.025em b}\kern-.08em
    T\kern-.1667em\lower.7ex\hbox{E}\kern-.125emX}}
\pgfplotsset{compat=1.14}

\newcommand{\Rplus}{\protect\hspace{-.1em}\protect\raisebox{.35ex}{\smaller{\smaller\textbf{+}}}}
\newcommand{\Cpp}{\mbox{C\Rplus\Rplus}\xspace}
\usepackage[utf8]{inputenc}
\usepackage{listings}
\usepackage[T1]{fontenc}
\usepackage{xcolor}
\usepackage{algorithm}
\usepackage{algorithmic,eqparbox,array}
\captionsetup[listing]{font={stretch=1.2}}

\definecolor{lightgray}{rgb}{.9,.9,.9}
\definecolor{darkgray}{rgb}{.4,.4,.4}
\definecolor{purple}{rgb}{0.65, 0.12, 0.82}

% auto labeling of sections
\let\origsection=\section
\renewcommand\section[1]{\origsection{#1}\label{sec:{#1}}}

\let\origsubsection=\subsection
\renewcommand\subsection[1]{\origsubsection{#1}\label{subsec:{#1}}}

\let\origsubsubsection=\subsubsection
\renewcommand\subsubsection[1]{\origsubsubsection{#1}\label{subsubsec:{#1}}}

\lstdefinelanguage{JavaScript}{
  keywords={typeof, new, false, catch, function, return, null, catch, switch, var, if, in, while, do, else, case, break},
%   keywordstyle=\color{blue}\bfseries,
  ndkeywords={class, export, boolean, throw, implements, import, this, type, interface, extends},
%   ndkeywordstyle=\color{darkgray}\bfseries,
%   identifierstyle=\color{black},
  sensitive=false,
  comment=[l]{//},
  morecomment=[s]{/*}{*/},
%   commentstyle=\color{purple}\ttfamily,
%   stringstyle=\color{red}\ttfamily,
  morestring=[b]',
  morestring=[b]"
}

% \lstset{
%   language=JavaScript,
% %   backgroundcolor=\color{lightgray},
% %   columns=fullflexible,
%   extendedchars=true,
%   basicstyle=\footnotesize\ttfamily,
%   showstringspaces=false,
%   showspaces=false,
% %   numbers=left,
%   numberstyle=\footnotesize,
%   numbersep=9pt,
%   tabsize=2,
%   breaklines=true,
%   showtabs=false,
%   captionpos=b
% }

\def\thesubsubsectiondis{\unskip\arabic{subsubsection})}

\renewcommand{\ttdefault}{pcr}
\lstset{
    aboveskip=0.2cm,
    stringstyle=\ttfamily,
    showstringspaces = false,
    basicstyle=\scriptsize\ttfamily,
    commentstyle=\color{gray!85},
    keywordstyle=\bfseries,
    ndkeywordstyle=\bfseries,
    identifierstyle=\ttfamily,
    numbers=none,
    % numbersep=5pt,
    % numberstyle=\tiny,
    % numberfirstline = false,
    breaklines=true,
    tabsize=2,
    % frame=L,
    rulecolor=\color{black},
    captionpos=t,
    % escapeinside=##
    breakatwhitespace=true,
    belowcaptionskip=10pt
    % xleftmargin=10pt
}

% \lstset{
%   basicstyle=\ttfamily,
%   columns=fullflexible,
%   frame=single,
%   breaklines=true,
%   basicstyle=\footnotesize, %or \tiny or \small etc.
% %   postbreak=\mbox{\textcolor{red}{$\hookrightarrow$}\space},
% }

\usepackage{hyperref}


\hypersetup{
    colorlinks=true,
    linkcolor=blue,
    filecolor=magenta,      
    urlcolor=brown,
    pdfpagemode=FullScreen,
    }
    
\usepackage{multicol}


\usepackage{tabulary}

\definecolor{Mycolor1}{HTML}{4C5866} %#4C5866
\definecolor{Mycolor2}{HTML}{A6B6D2} %#A6B6D2
\definecolor{Mycolor3}{HTML}{C1D4E8} %#C1D4E8
\definecolor{Mycolor4}{HTML}{F7F7FE} %#F7F7FE
\definecolor{Mycolor5}{HTML}{d8f3dc} %#d8f3dc
\definecolor{Mycolor6}{HTML}{EC6A5C} %#EC6A5C
\definecolor{Mycolor7}{HTML}{EDE992} %#EDE992


% Switch for algorithmic
\newcommand{\Switch}[1]{\STATE \textbf{switch} (#1)}
\newcommand{\EndSwitch}{\STATE \textbf{end switch}}
\newcommand{\Case}[1]{\STATE \textbf{case} #1\textbf{:} \begin{ALC@g}}
\newcommand{\EndCase}{\end{ALC@g}}
\newcommand{\Caseline}[1]{\STATE \textbf{case} #1\textbf{:} }
\newcommand{\Default}{\STATE \textbf{default:} \begin{ALC@g}}
\newcommand{\EndDefault}{\end{ALC@g}}
\newcommand{\DefaultLine}[1]{\STATE \textbf{default:} }