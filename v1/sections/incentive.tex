\section{Incentive}

Proper incentive is essential for a decentralized network to provide a satisfactory level of service. To ensure that the incentive scheme is transparent, sustainable, and fair to all network participants, DAO is put in charge to dynamically adjust incentives when required.

Current successful incentive schemes are predominantly found in blockchain-based networks: all nodes are required to verify transactions to be rewarded by ``transaction fees''. The transaction-based model is proven to be functional in mostly financial-related fields, but is expected to struggle under other settings. Take social media as an example, it is unfeasible to charge users for interactions performed. Nor should developers be paying for the cost of maintaining a decentralized network.

\subsection{Incentivization}

The RSS3 Network, on the other hand, will be rewarding network participants with the profit of the network generated from advertising, value-added services, social economic activities, etc.

\begin{itemize}
    \item \textbf{Stage 1: System Incentivization} At the beginning, the network is rewarded by the system to encourage adoption. The system reward gradually decreases and is inversely proportional to network adoption.
    
    \item \textbf{Stage 2: Hybrid Incentivization} As network adoption increases, profits are expected to be generated by activities such as advertisements, value-added services and other related economic activities. The proceeds will be distributed in the form of network rewards, which also offsets the system rewards.

    \item \textbf{Stage 3: Self-Sustained Incentivization}: All rewards are now fully transitioned into network-generated rewards, the system will no longer incentivize the network. 
\end{itemize}

Incentives will be distributed to network participants including node hosts, developers, content creators, special contributors and DAO.

\subsection{Staking and Slashing}

Upon election, GIs and SNs are required to join the staking pool governed by DAO. Staking demonstrates the participant's commitment toward maintaining the network.

Consequently, slashing as a means to maintain the network's stability occurs when malicious or erroneous behaviors are detected, in addition to situations described in Section \ref{subsec:{Global Indexer (GI)}} and \ref{subsec:{Serving Node (SN)}}.

\subsection{Incentive Pool}

An incentive pool $\mathcal{I}$ is introduced to dynamically balance the incentives distributed to network participants, see Eqn. \ref{eqn:incentive_0}, \ref{eqn:incentive_1}, and \ref{eqn:incentive_2}, where the proportions set $\Phi=\{\alpha, \beta, \gamma, \delta, \epsilon\}$) is decided by DAO; The participants set $P = \{p_{nod}, p_{dev}, p_{cre}, p_{con}, p_{dao}\}$ includes node hosts, developers, content creators, special contributors and DAO respectively. This will ultimately foster a self-adjusting market and avoid skewed incentives in the distribution: Incentive increases as the demand for a role increases and vice versa. A proper allocation of the incentive pool greatly motivates the network participants in improving all aspects of the network.

\begin{gather}
    \emph{sum}(\alpha,\beta,\gamma,\delta,\epsilon) = 1 \label{eqn:incentive_0}\\
    \begin{cases}
        p_{nod} = \alpha \mathcal{I}\\
        p_{dev} = \beta \mathcal{I}\\
        p_{cre} = \gamma \mathcal{I}\\
        p_{con} = \delta \mathcal{I}\\
        p_{dao} = \epsilon \mathcal{I}
    \end{cases} \label{eqn:incentive_1}\\
    \mathcal{I} = p_{nod} + p_{dev} + p_{cre} + p_{con} + p_{dao} \label{eqn:incentive_2}
\end{gather}