\section{Introduction} 

The original RSS Standard is open, neutral, and decentralized in nature. In the Web 2.0 era, the adoption of RSS dropped significantly as centralized platforms gained traction. However, a centralized platform, where a single trusted party dominates the network, inevitably results in problems such as privacy infringement, profit misappropriation, censorship, algorithm abuse, and data monopoly. Some innovative solutions have been proposed to tackle these problems: 1) Federated networks enable users to choose a trusted center while still supporting communications among centers; 2) Blockchain-based networks allow user data to be distributed across all nodes. To some extent, these solutions manage to solve certain problems, but with the limitations of flexibility, efficiency, and extensibility.

As the Web moves toward more openness and modularization, the pipeline of information flow is evolving to include four decentralized layers including creation, storage, distribution, and rendering. Various protocols or modules are expected to support an interoperable communication that can be fully controlled by users. Existing solutions do not form a fully decentralized information distribution standard that is urgently demanded by the Web.

In this paper, we propose the RSS3 Standard, a next-generation feed standard that enables efficient and decentralized information distribution with flexibility, efficiency, and extensibility. The standard ensures that the network will be financially self-sustained with redundancy and fault tolerance. We then introduce the RSS3 Protocol with a sophisticated network structure to implement the standard. Multiple measures are implemented to ensure the network's stability and security. 
