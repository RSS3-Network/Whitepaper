\section{glossary}

This section includes the glossary Table~\ref{table:glossary} contains a list of terms used throughout the paper.

{
\renewcommand{\arraystretch}{1.5}
\begin{table}[!ht]
	\rowcolors{2}{}{Mycolor1!10}
    \resizebox{\columnwidth}{!}{
		\begin{tabulary}{\columnwidth}{|l|p{8cm}|}
			\hline
			\rowcolor{Mycolor1!30} \multicolumn{1}{|l|}{\textbf{Term}}&\multicolumn{1}{|l|}{\textbf{Definition}}\\
			\hline
			% \atlas{Shark} & A Shark refers to a collection of data generated by one cyber existence.\\
			% \hline
    		DAO & Decentralized Autonomous Organization, an automated decision-making organization that is agreed by the network and is not influenced by a central authority. DAO oversees all important decisions. \\
    		\hline
    		SN & A serving node, which hosts RSS3 Files and responds to file-related requests. DAO sets a limit of files that a single SN can serve.  \\
    		\hline
    		Subgroup & A group of SNs, which consists of a number of SNs. DAO sets a limit of SNs that a single subgroup can have.  \\
    		\hline
    		GI & A global indexer, which orchestrates subgroups, routes client requests, and maintains the network capability. DAO sets the incentive scheme. \\
    		\hline
    		RN & A relay node, as part of a GI to assist in routing. \\
    		\hline
    		Epoch & A reference point in time where multiple events are scheduled to be triggered. \\
    		\hline
    		ER & Epoch round, a period of time between two epoch points where the network operates without any planned events from DAO and GIs. DAO sets the ER interval. \\
    		\hline
    		SDG & Scalable Dynamic Grouping, the strategy used by GIs to orchestrate subgroups that will be executed at the epoch.  \\
			\hline
			MOC & Minimal operational capability, the required number of GIs or SNs in a subgroup, to provide trustworthy services. \\
			\hline
			HBC & Heartbeat checking, a strategy used by nodes to report their working status. \\
			\hline
		\end{tabulary}
		
	}
	\caption{A glossary table that contains the terms used throughout the paper.}
	\label{table:glossary}
\end{table}
}