\documentclass[conference]{IEEEtran}

% The preceding line is only needed to identify funding in the first footnote. If that is unneeded, please comment it out.
\renewcommand\IEEEkeywordsname{Keywords}

\usepackage{caption}
\usepackage{url}
\usepackage{titletoc}
\usepackage{graphicx}
\usepackage[utf8]{inputenc}
\usepackage[T1]{fontenc}
\usepackage[table]{xcolor}
\usepackage{tabulary}

\usepackage{titlesec}
\def\thesubsubsectiondis{\unskip\arabic{subsubsection})}
\titlespacing*{\subsubsection}{0pt}{10pt plus 4pt minus 2pt}{4pt plus 2pt minus 2pt}

\usepackage{hyperref}
\hypersetup{
    colorlinks=true,
    linkcolor=blue,
    urlcolor=brown,
    }

\usepackage[nopostdot,nonumberlist,automake,symbols]{glossaries-extra}

\usepackage{ifthen}

\setabbreviationstyle{long-short}
\setglossarystyle{altlist}

\makeglossaries

% % only show link on first use
\GlsXtrEnableEntryUnitCounting{general}{0}{chapter}
\renewcommand*{\glslinkcheckfirsthyperhook}{%
    \ifnum\glsentrycurrcount\glslabel>0
        \setkeys{glslink}{hyper=false}%
    \fi
}

% #1 - short name: OIL
% #2 - math symbol
% #3 - full name: Open Information Layer
% #4 - description

\newcommand{\newdualentry}[4]{%
    \ifthenelse{\equal{#2}{}}%
    {
        % If #2 (the math symbol) is empty
        \newglossaryentry{#1}{%
            name={#3 (#1)},
            text={#1},
            short={#1},
            long={#3},
            first={#3 (#1)},
            firstplural={#3s (#1s)},
            description={#4}
        }
    }{
        % If #2 is provided
        \newglossaryentry{#1}{%
            name={#3 (\ensuremath{#2})},
            text={#1},
            short={\ensuremath{#2}},
            long={#3},
            first={#3 (\ensuremath{#2})},
            firstplural={#3s (#1s)},
            description={#4},
            symbol={\ensuremath{#2}}
        }
    }
}


\newdualentry{OW}{}{Open Web}{The next-generation Internet where information flows openly without any restrictions, as it is supposed to be.}

\newdualentry{OI}{}{Open Information}{Information that is typically found across various types of networks, including decentralized, federated, and centralized networks that allow permissionless access.}

\newdualentry{OIL}{}{Open Information Layer}{A conceptual layer where information flows openly without any restrictions.}

\newdualentry{DSL}{}{Data Sub-layer}{A decentralized network where the Open Information flows from its source to its destination.}

\newdualentry{UMS}{}{Unified Metadata Schemas}{A unified set of data structures for interoperability.}

\newdualentry{SN}{}{Serving Node}{A Data Sub-layer component that indexes, cleans, stores, and ultimately serves the Open Information to the end users.}

\newdualentry{GI}{}{Global Indexer}{A Data Sub-layer component that facilitates coordination among Serving Nodes and engages with the Value Sub-layer.}

\newdualentry{VSL}{}{Value Sub-layer}{A blockchain where the value created by Open Information activities is recorded and distributed.}

\newdualentry{PDS}{}{Permissionless Data Source}{A repository of data that can be accessed without the need for authorization or authentication.}

\newdualentry{NO}{}{Node Operator}{An individual or organization that operates a Serving Node.}

\newdualentry{PGP}{P_p}{Public Good Pool}{A collective pool of staked \$RSS3 that is used to improve the RSS3 Network by assigning trust to Public Good Nodes.}

\newdualentry{OP}{P_o}{Operating Pool}{A pool of \$RSS3 that consists of 1) Fees collected from serving Data Sub-layer requests; 2) Network Rewards allocated based on the Node's work.}

\newdualentry{SP}{P_s}{Staking Pool}{A pool of staked \$RSS3 that is used to improve the RSS3 Network by assigning trust to Normal Nodes.}

\newdualentry{R3N}{}{RSS3 Network}{A decentralized network that is formed by a DSL and a VSL.}

\usepackage[nameinlink]{cleveref}

\begin{document}

\title{RSS3: The Open Information Layer}

\author{Natural Selection Labs}
\maketitle

\tableofcontents

\thispagestyle{plain}
\pagestyle{plain}
\pagenumbering{arabic}

\begin{abstract}

    Inspired by the original RSS Standard, this paper presents RSS3, the \glsfmtlong{OIL} for the \glsfmtlong{OW}. The paper serves as an enhanced version of our initial whitepaper titled ``RSS3: A Next-Generation Feed Standard.'' Following the release of our initial whitepaper, we have adhered to its proposed architecture to conduct experiments and advance the development of the RSS3 Network. The Network has transformed into what is now known as the \glsfmtlong{OIL}, reflecting the evolving dynamics of the \glsfmtlong{OW}. This paper summarizes our research and development output since then, providing insights into RSS3's vision and its decentralization architecture. Finally, we present the Network's tokenomics and governance model, and discuss the future of RSS3.

\end{abstract}

\section{Introduction} 

RSS3 is the \glsfmtlong{OIL}, structuring \glsfmtlong{OI} for social, search, and AI. The \gls{OIL} is a conceptual layer where information flows openly without any restrictions, as it is supposed to be.

It is RSS3's mission to construct the \glsfmtlong{OW} by enhancing the free flow of \glsfmtlong{OI}.



\section{\glsfmtlong{R3N}}

The \glsfmtlong{R3N} is a decentralized network that is formed by two sub-layers: the \glsfirst{DSL} and the \glsfirst{VSL}

\gls{OI} is typically found across various types of networks, including decentralized, federated, and centralized networks that allow permissionless access.
The \gls{DSL} is responsible for indexing and structuring \gls{OI} for interoperability. 
This is achieved by introducting a crucial standard, known as the \gls{UMS}, see \Cref{subsec:UMS}, enabling network-agnostic applications to be built on top of the \gls{DSL}.
The \gls{DSL} then leverages the \gls{VSL}, see \Cref{sec:VSL}, to build an ownership economy on the \gls{OW}.

\$RSS3 is the Network's native utility token. It is used to pay query fees, operate nodes, participate in staking, and engage in various network activities.

\section{\glsfmtlong{DSL}}
\label{sec:DSL}

The \glsfirst{DSL} is responsible for \glsfmtlong{OI} life cycle management, which includes indexing, transformation, storage, dissemination, and consumption \cite{nationalinstituteofstandardsandtechnology2016Information}. The \gls{DSL} is formed by two components (see \cref{subsec:SN} and \cref{subsec:GI}), and uses the \gls{UMS} (see \cref{subsec:UMS}) to structure the information for applications in social, search, AI and beyond.

\subsection{\glsfmtfull{SN}}
\label{subsec:SN}

An \gls{SN}, also know as an RSS3 Node, is responsible for indexing, transforming, storing, and ultimately serving the \glsfmtlong{OI} to the end users. Each \gls{SN} operates a number of workers that index and structure information from \gls{PDS}, stores the information, and provides interfaces for access. Workers are community-maintained ``rules'' that define how information is indexed and transformed into the \gls{UMS} format. 

Since each \gls{SN} is independent, it is possible for different \glspl{SN} to deploy different workers that cover different \glspl{PDS}.

This design enables node operation to be flexible, accessible and affordable, in turn, offering a high degree of decentralization and robustness.

\subsection{\glsfmtfull{GI}}
\label{subsec:GI}

A \gls{GI} is responsible for facilitating coordination among \glspl{SN} and engaging with the \gls{VSL}, and performs the following functions:

\subsubsection{Performance Assurance} A GI acts as a load balancer and query router for end users to retrieve information from \glspl{SN}. The unique architecture of the \gls{DSL} demands \glspl{GI} to be equipped with more computational capabilities, in order to work out the optimal route for end users to retrieve specific information from \gls{SN}, and frequently from a group of \glspl{SN} simultaneously.

\subsubsection{Quality Assurance} A GI acts as a supervisor for \glspl{SN} to ensure the quality of service. With the \gls{DSL} being a permissionless sub-layer, the quality needs to be maintained strictly to ensure \glsfmtlong{R3N}'s robustness and reliability.
A \gls{GI} monitors the quality of \glspl{SN}, and slashes the \gls{SN} if it fails to meet the requirements.

\subsubsection{Proof-on-Chain} A GI keeps track of the work and slash records of \glspl{SN}, and submits them to the \gls{VSL} for settlement and reward allocation.

\subsection{\glsfmtfull{UMS}}
\label{subsec:UMS}

Open Information, indexed from multiple \glspl{PDS}, is structured by \glspl{SN} into the \gls{UMS} format for interoperability.

\glspl{PDS} use different data structures, within a \gls{PDS}, there might be multiple products, services and protocols that leverage a different data structure to suit their needs. This means limited interoperability, and developers need to look into each and every data structure, when it comes to building. This lack of standardization means developers must investigate each unique structure individually when building applications, which is not scalable.

The \gls{UMS} addresses this issue by offering a unified set of data structures that serve as an abstraction. This abstraction simplifies the integration process, making it more manageable and scalable for developers to work with data across various data sources.

For the complete set of the \gls{UMS}, refer to \url{https://docs.rss3.io/docs/unified-metadata-schemas}.



\section{\glsfmtlong{VSL}}
\label{sec:VSL}

The \glsfirst{VSL}, commonly referred to as the RSS3 Chain, is an Ethereum Layer 2 blockchain built with OP Stack uisng Celestia as the data availability layer.
It is responsible for handling value derived from \glsfmtlong{OI} activities and applications, establishing a healthy ownership economy for the Network.

In this section, we focus on the intentions behind the \gls{VSL}'s incentive mechanism, which is designed to promote stable Node Operations to maintain the Network, and to encourage network participants to secure the Network via staking \$RSS3.
We introduce the detailed tokenomics separately in \Cref{sec:tokenomics}.

The \glsfmtlong{R3N} allocates a portion of \$RSS3 total supply to incentivize network participants, referred to as the \glsfirst{NR}, 
are allocated into reward pools: the \glsfirst{OP} and the \glsfirst{SP} for Normal Nodes, or the \glsfirst{PGP} for Public Good Nodes.
See \Cref{fig:network-rewards} for an illustration and \Cref{subsec:reward_pools} for details on Reward Pools.
The calculation of \glsfmtlong{NR} are described in \Cref{sec:tokenomics}.

\subsection{Node Operation}
\Glsfmtlong{NO}s are incentivized to operate and maintain the Network by receiving \$RSS3 as rewards.
\begin{enumerate}
    \item Anyone can become a \glsfmtlong{NO} to launch an RSS3 Node and join the RSS3 Network without requiring prior permission.
    \item A \glsfmtlong{NO} has the ability to configure Node's coverage, which directly influences the Node's capability to respond to various types of requests. A broader coverage means more computational resources are required, and a higher chance of receiving requests.
    \item A Node can be operated in either a Normal mode or a Public Good mode. A Normal Node is eligible for \glsfmtlong{NR}, but requires a deposit of \$RSS3. A Public Good Node is ineligible for \glsfmtlong{NR}, but requires no deposit.
    \item A Normal Node has a corresponding \operationPool\ and a \stakingPool. All Public Good Nodes collectively share a single \publicGoodPool.
\end{enumerate}

\subsection{Node Staking}
Network participants are incentivized to stake \$RSS3 to secure the Network by receiving \$RSS3 as rewards.
\begin{enumerate}
    \item A Normal Node accepts staking into its Reward Pool, the amount of staked \$RSS3 signifies its quality. Higher quality Nodes handle more requests.
    \item A Public Good Node does not have a Reward Pool and does not participate in any form of incentivization. Staking into a \glsfmtlong{PGP} is accepted, and the stakers can assign their trust to any Public Good Node. Higher trust Nodes handle more requests.
\end{enumerate}

{
\renewcommand{\arraystretch}{1.5}
\begin{table*}[h]
    \resizebox{\textwidth}{!}{
        \begin{tabulary}{\textwidth}{|p{6cm}|p{5cm}|p{5cm}|}
            \hline
            & \textbf{Node in Normal Mode} & \textbf{Node in Public Good mode} \\ \hline
            Who can operate? & Anyone & Anyone \\ \hline
            Can \glsfmtlong{NO} specify the coverage? & Yes & Yes \\ \hline
            Is a deposit required? & Yes & No \\ \hline
            Is the deposit considered as staking, making it eligible for rewards from its own \stakingPool? & No & N/A \\ \hline
            Will the Node be slashed? & Yes, its deposit and \stakingPool will be slashed. A Node may be demoted to receive fewer requests. & No, but a Node may be demoted to receive fewer requests. \\ \hline
            Does the Node accept staking? & Yes. The staked tokens go to the Node’s \stakingPool. RSS3-X (X being the Node’s name) Chips are issued to the stakers after staking. & No, as such a Node does not have a \glsfmtlong{SP}. Instead, stakers stake to a \glsfmtlong{PGP}. RSS3-Public Good Chips are issued to the stakers after staking. \\ \hline
            Can \glsfmtlong{NO} set an \glsfirst{Tax}? & Yes & No, a universal tax is determined by the Network. \\ \hline
            Does it have an \glsfmtlong{OP}? & Yes & No, operator rewards go to [X] \\ \hline
            Does it have a \glsfmtlong{SP}? & Yes & No, but a \glsfmtlong{PGP} with a universal incentive rate. \\ \hline
        \end{tabulary}
    }
    \caption{Comparison of two Node operation modes.}
    \label{table:node_modes}
\end{table*}
}

\subsection{Reward Pools}
\label{subsec:reward_pools}

This section introduces the three reward pools: the \glsfirst{OP}, the \glsfirst{SP}, and the \glsfirst{PGP}. See \Cref{fig:network-rewards} for an illustration.

{
\begin{figure}[tb!]
    \centering
    \includegraphics[width=0.9\columnwidth]{figures/network-rewards.png}
    \caption{RSS3 \glsfmtlong{NR} distribution.
    The \glsfmtlong{NR} are allocated into two reward pools: the \glsfirst{OP} and the \glsfirst{SP} for Normal Nodes, or the \glsfirst{PGP} for Public Good Nodes.
    See \Cref{subsec:reward_pools} for details.}
    \label{fig:network-rewards}
\end{figure}
}

\subsubsection{\glsfmtlong{OP} (\operationPool)}
\label{subsubsec:operation_pool}

An \glsfirst{OP} is used to store tokens that are allocated to a Normal Node from three sources: 1) the \gls{Fee} collected from requests served on the \gls{DSL}, denoted \work; 3) the \glsfmtlong{NR} allocated based on the Node’s work; 3) the \gls{Tax} collected from the Node's \stakingPool.

The allocation of \glsfmtlong{NR} into a Node's \operationPool\ at the end of each epoch, is determined by the \glsfmtlong{N} (\work), in proportion to the total number of requests served on the \gls{DSL}.

The \glsfmtlong{NO} can set a tax rate, \tax, which is applied to its \stakingPool.
The tax applies to the \glsfmtlong{NR} allocated to the Node's \stakingPool, not the staked tokens (REP-1: Chip Redemption Tax).

Only the corresponding \glsfmtlong{NO} can withdraw tokens from its \operationPool, and the withdrawal is subject to a waiting period imposed by the Network.

\subsubsection{\glsfmtlong{SP} (\stakingPool)}
\label{subsubsec:staking_pool}

A \glsfirst{SP} is used to store staked tokens for a Normal Node. Network participants can stake tokens into a Normal Node's \stakingPool\ to increase the Node's chance to receive requests on the \gls{DSL}.

The allocation of \glsfmtlong{NR} into a Node's \stakingPool\ at the end of each epoch, is determined by the size of the Node's \stakingPool, in proportion to the total staked tokens on the \gls{VSL}.
A tax is then applied to the received Rewards, with the rate set by its \glsfmtlong{NO}.

\subsubsection{\glsfmtlong{PGP} (\publicGoodPool)}

A \glsfirst{PGP} is a unique reward pool that is shared by all Public Good Nodes.

As Public Good Nodes do not have their own \stakingPool, network participants stake into the \publicGoodPool\ to assign their trust to any Public Good Node.


\section{Tokenomics}
\label{sec:tokenomics}

In this seciton, we introduce the detailed tokenomics of the \glsfmtlong{R3N}. We present the \glsfmtlong{NR}'s calculation and distribution formulas, and the slashing mechanism employed to enforce network security and stability.

\subsection{\glsfmtfull{NR}}
IN \Cref{sec:VSL}, we describe the intentions behind the \gls{VSL}'s incentive mechanism, here we introduce the detailed \glsfmtlong{NR} calculation and distribution formulas separately.

The \glsfmtlong{NR} \networkReward\ consists of three parts: 
\begin{equation}
    \label{eq:network_rewards}
    \networkReward = (\operationReward + \stakingReward) + \publicGoodReward
\end{equation}

See \Cref{fig:network-rewards} for an illustration.

\subsubsection{\glsfmtfull{OR}}
To encourage Normal Nodes to operate and maintain the Network, \operationReward\ is allocated to a Node's \operationPool\ in proportion to the fees collected from its work contribution.

\begin{equation}
    \label{eq:operation_weight}
    \tilde{\fee}_\nodeAtEpoch = \log_{2}(\frac{\fee_\nodeAtEpoch}{\sum_{x=0}^{\infty} \fee_{x, \epoch}} + 1) * G
\end{equation}

$\tilde{\fee}_\nodeAtEpoch$ denotes the normalized work contribution for a given Node \node, at the end of a given epoch $\epoch$. $G$ is a constant equal to $\ln(2) \approx 0.693147$ used to offset the effect of replacing $\ln$ with $\log_2$, as the former is more costly in terms of gas when it comes to on-chain computation.

\begin{equation}
    \label{eq:operation_rewards}
    \networkReward_{\operation|\node, \epoch} = \frac{\tilde{\fee}_{\node, \epoch}}{\sum_{x=0}^{\infty} \tilde{\fee}_{x, \epoch}} * \networkReward_{\operation, \epoch}
\end{equation}

$\networkReward_{\operation|\node, \epoch}$ therefore denotes the Operation Rewards for a given Node \node, at the end of a given epoch $\epoch$.

\subsubsection{\glsfmtfull{SR}}

To encourage participation from all network participants to increase the Network's reliability, \stakingReward\ is allocated to a Node's \stakingPool\ in proportion to the amount of staked tokens in the entire Network.


\subsection{Slashing}

\section{Conclusion} 

\textbf{At the heart of Natural Selection Labs, we firmly believe in the freedom of information allocation: No organizations or authorities shall prohibit the free exercise of the right of people to create, store, and distribute their information.}


\printglossary[type=main,title=Glossary, toctitle=Glossary]



\bibliographystyle{plain}
\bibliography{references}

\end{document}

