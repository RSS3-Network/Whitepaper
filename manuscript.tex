\documentclass[conference]{IEEEtran}

% The preceding line is only needed to identify funding in the first footnote. If that is unneeded, please comment it out.
\renewcommand\IEEEkeywordsname{Keywords}

\usepackage{caption}
\usepackage{url}
\usepackage{titletoc}
\usepackage{graphicx}
\usepackage[utf8]{inputenc}
\usepackage[T1]{fontenc}
\usepackage[table]{xcolor}

\def\thesubsubsectiondis{\unskip\arabic{subsubsection})}

\usepackage{hyperref}
\hypersetup{
    colorlinks=true,
    linkcolor=blue,
    urlcolor=brown,
    }

% import math notations
\newcommand{\epoch}{\ensuremath{\epsilon}}

\newcommand{\work}{\ensuremath{W}}
\newcommand{\fee}{\ensuremath{F}}
\newcommand{\deposit}{\ensuremath{D}}

\newcommand{\node}{\ensuremath{N}}
\newcommand{\publicGoodNode}{\ensuremath{\node_p}}
\newcommand{\staking}{\ensuremath{s}}
\newcommand{\chip}{\ensuremath{C}}
\newcommand{\trust}{\ensuremath{t}}
\newcommand{\operation}{\ensuremath{o}}

\newcommand{\nodeAtEpoch}{\ensuremath{{\node,\epoch}}}


\newcommand{\pool}{\ensuremath{P}}

\newcommand{\stakingPool}{\ensuremath{\pool_\staking}}

\newcommand{\operationPool}{\ensuremath{\pool_\operation}}

\newcommand{\publicGood}{\ensuremath{p}}
\newcommand{\publicGoodPool}{\ensuremath{\pool_\publicGood}}

\newcommand{\networkReward}{\ensuremath{R}}

\newcommand{\stakingReward}{\ensuremath{\networkReward_\staking}}
\newcommand{\operationReward}{\ensuremath{\networkReward_\operation}}
\newcommand{\trustReward}{\ensuremath{\networkReward_\trust}}

\newcommand{\tax}{\ensuremath{T}}
\newcommand{\taxCap}{\ensuremath{c}}
\newcommand{\taxRate}{\ensuremath{\tau}}

\newcommand{\reliabilityScore}{\ensuremath{\sigma}}


\usepackage[nopostdot,automake]{glossaries-extra}

\usepackage{ifthen}

\setabbreviationstyle{long-short}
\setglossarystyle{altlist}

\makeglossaries
% only show link on first use
\GlsXtrEnableEntryUnitCounting{general}{0}{chapter}
\renewcommand*{\glslinkcheckfirsthyperhook}{%
    \ifnum\glsentrycurrcount\glslabel>0
        \setkeys{glslink}{hyper=false}%
    \fi
}

% #1 - short name: OIL
% #2 - math symbol
% #3 - full name: Open Information Layer
% #4 - description

\newcommand{\newdualentry}[4]{%
    \ifthenelse{\equal{#2}{}}%
    {
        % If #2 (the math symbol) is empty
        \newglossaryentry{#1}{%
            name={#3 - #1},
            text={#1},
            short={#1},
            long={#3},
            first={#3 (#1)},
            firstplural={#3s (#1s)},
            description={#4},
            sort={#1}
        }
    }{
        % If #2 is provided
        \newglossaryentry{#1}{%
            name={#3 - \ensuremath{#2}},
            text={#1},
            short={\ensuremath{#2}},
            long={#3},
            first={#3 (\ensuremath{#2})},
            firstplural={#3s (#1s)},
            description={#4},
            symbol={\ensuremath{#2}},
            sort={#1}
        }
    }
}

\newdualentry{OW}{}{Open Web}{The next-generation Internet where information flows openly without any restrictions, as it is supposed to be.}

\newdualentry{OI}{}{Open Information}{Information that is typically found across various types of networks, including decentralized, federated, and centralized networks that allow permissionless access.}

\newdualentry{OIL}{}{Open Information Layer}{A decentralized and permissionless information layer where information flows openly without any restrictions.}

\newdualentry{DSL}{}{Data Sublayer}{A decentralized network where the Open Information flows from its source to its destination.}

\newdualentry{Protocol}{}{RSS3 Protocol}{A unified set of data structures for interoperability.}

\newdualentry{Node}{\node}{Node}{A Data Sublayer component that indexes, cleans, stores, and ultimately serves the Open Information to the end users. Denoted as \node\ when it is in Normal mode, and \publicGoodNode\ when it is in Public Good mode.}

\newdualentry{SNP}{\publicGoodNode}{Node (Public Good Mode)}{A \gls{Node} that operates for the purpose of supporting Public Goods and strenthening the Network, it does not receive any \glsfmtlong{Fee} or \glsfmtlong{NR}.}

\newdualentry{RS}{\reliabilityScore}{Reliability Score}{A score used to determine the allocation of requests to \glsfmtlong{Node}s.}

\newdualentry{GI}{}{Global Indexer}{A \glsfmtlong{DSL} component that facilitates coordination among \glsfmtlong{Node}s and engages with the \glsfmtlong{VSL}.}

\newdualentry{VSL}{}{Value Sublayer}{A blockchain where the value created by Open Information activities is recorded and distributed.}

\newdualentry{PDS}{}{Permissionless Data Source}{A repository of data that can be accessed without the need for authorization or authentication.}

\newdualentry{NR}{\networkReward}{Network Rewards}{Tokens allocated by the RSS3 Network to incentivize network participants.}

\newdualentry{OR}{\operationReward}{Operation Rewards}{Tokens allocated to \glsfmtlong{OP} by the RSS3 Network to incentivize Node operation.}

\newdualentry{SR}{\stakingReward}{Staking Rewards}{Tokens allocated to \glsfmtlong{SP} by the RSS3 Network to incentivize network participation.}

\newdualentry{TR}{\trustReward}{Trust Rewards}{Tokens allocated to \glsfmtlong{PGP} by the RSS3 Network to incentivize network participation and support Nodes for Public Goods provision.}

\newdualentry{NO}{}{Node Operator}{An individual or organization that operates a Node.}

\newdualentry{PGP}{\publicGoodPool}{Public Good Pool}{A collective pool of staked \$RSS3 that is used to improve the RSS3 Network by assigning trust to Public Good Nodes.}

\newdualentry{OP}{\operationPool}{Operation Pool}{A pool of \$RSS3 that consists of 1) Fees collected from serving Data Sublayer requests; 2) \glsfmtlong{NR} allocated based on the Node's work; 3) Tax collected from its Staking Pool.}

\newdualentry{SP}{\stakingPool}{Staking Pool}{A pool of staked \$RSS3 that is used to improve the RSS3 Network by assigning trust to Normal Nodes.}

\newdualentry{R3N}{}{RSS3 Network}{A decentralized network that is formed by a DSL and a VSL.}

\newdualentry{Epoch}{\epoch}{Epoch}{A period of time used as a reference to measure the \glsfmtlong{R3N}'s operation.}

\newdualentry{N}{\work}{number of requests served on the \glsfmtlong{DSL}}{The metric that reflects the Node's service and is used to determine the allocation of \glsfmtlong{NR} into a Node's \glsfmtlong{OP}.}

\newdualentry{Tax}{\tax}{Operation Tax}{A tax collected from the \glsfmtlong{NR} that are allocated to a Node's \glsfmtlong{SP}, by its \glsfmtlong{OP}.}

\newdualentry{Deposit}{\deposit}{Deposit}{Tokens required to operate a \glsfmtlong{Node}.}

\newdualentry{Fee}{\fee}{Request Fees}{Fees paid to \glspl{Node} for delivering \glsfmtlong{OI} from its \glsfmtlong{PDS} to the requesters.}

\newdualentry{Chip}{\chip}{Chip}{An ERC-721 Non-Fungible Tokens (NFTs) representing a network participant's stake in a particular Node.}

\newdualentry{Slashing}{}{Slashing}{A penalty imposed on a Node for misconduct.}

\newdualentry{REP}{}{RSS3 Evolution Proposal}{The proposal to change the RSS3 Network's parameters, processes and rules.}


\begin{document}

\title{RSS3: The Open Information Layer}

\author{Natural Selection Labs}
\maketitle

\thispagestyle{plain}
\pagestyle{plain}
\pagenumbering{arabic}

\begin{abstract}

Inspired by the original RSS Standard, this paper presents RSS3, \acrlong{OIL} for the \acrlong{OW}. The paper serves as an enhanced version of our initial whitepaper titled ``RSS3: A Next-Generation Feed Standard.'' Following the release of our initial whitepaper, we have adhered to its proposed architecture to conduct experiments and advance the development of the RSS3 Network. The Network has transformed into what is now known as the \acrlong{OIL}, reflecting the evolving dynamics of the \acrlong{OW}. This paper summarizes our research and development progress since then, providing insights into RSS3's vision and its decentralization architecture.

\end{abstract}

\section{Introduction}

RSS3 is the \glsfmtlong{OIL}, structuring \glsfmtlong{OI} for social, search, and AI.
The \gls{OIL} is a decentralized and permissionless layer where information flows openly without any restrictions, as it is supposed to be.

It is RSS3's mission to construct the \glsfmtlong{OW} by enhancing the free flow of \glsfmtlong{OI}.



\section{Open Information Layer}

The \gls{OIL} is a conceptual layer that is formed by two sub-layers: the \gls{DSL} and the \gls{VSL}. Information from permissonless data sources on the \gls{OIL} flows openly without any restrictions.


\section{Data Sub-layer}
\label{sec:DSL}

The \gls{DSL} is responsible for information life cycle management, which includes indexing, transformation, storage, dissemination, and consumption \cite{nationalinstituteofstandardsandtechnology2016Information}. The \gls{DSL} is formed by two components (see \cref{subsec:SN} and \cref{subsec:GI}), and uses the \gls{UMS} (see \cref{subsec:UMS}) to structure the information.

\subsection{\glsfirst{SN}}
\label{subsec:SN}

An \gls{SN} is responsible for indexing, cleaning, storing, and ultimately serving the Open Information to the end users. Each \gls{SN} operates a number of workers that index and structure information from \gls{PDS}, stores the information, and provides interfaces for access.

\subsection{\glsfirst{GI}}
\label{subsec:GI}

An \gls{GI} is responsible for facilitating coordination among \glspl{SN} and engaging with the \gls{VSL}, and performs the following functions:
\begin{enumerate}
    \item A load balancer and query router for end users to retrieve information from \glspl{SN}.
    \item A supervisor for \glspl{SN} to ensure the quality of service.
    \item A settler for submitting work and slash records to the \gls{VSL}.
\end{enumerate}


\subsection{\glsfirst{UMS}}
\label{subsec:UMS}

Open Information, indexed from multiple \glspl{PDS}, is structured by \glspl{SN} into the \gls{UMS} format for interoperability.

\glspl{PDS} use different data structures, within a \gls{PDS}, there might be multiple products, services and protocols that leverage a different data structure to suit their needs. This means limited interoperability, and developers need to look into each and every data structure, when it comes to building. This lack of standardization means developers must investigate each unique structure individually when building applications, which is not scalable.

The \gls{UMS} addresses this issue by offering a unified set of data structures that serve as an abstraction. This abstraction simplifies the integration process, making it more manageable and scalable for developers to work with data across various data sources.

For the complete set of the \gls{UMS}, refer to \url{https://docs.rss3.io/docs/unified-metadata-schemas}.



\section{Value Sub-layer}


\section{Incentive}

\subsection{Incentivization}

The RSS3 Network, on the other hand, will be rewarding network participants with the profit of the network generated from advertising, value-added services, social economic activities, etc.

\subsection{Staking and Slashing}

\subsection{Incentive Pool}

\subsubsection{Operator Pool}

\subsubsection{Reward Pool}


\section{Scalability}


\section{Conclusion} 

\textbf{At the heart of Natural Selection Labs, we firmly believe in the freedom of information distribution: No organizations or authorities shall prohibit the free exercise of the right of people to create, store, and distribute their information.}


\printglossary[type=main,title=Glossary, toctitle=Glossary]



% \bibliographystyle{plain}
% \bibliography{references}

\end{document}

