\documentclass[conference]{IEEEtran}

% The preceding line is only needed to identify funding in the first footnote. If that is unneeded, please comment it out.
\renewcommand\IEEEkeywordsname{Keywords}

\usepackage{caption}
\usepackage{url}
\usepackage{titletoc}
\usepackage{graphicx}
\usepackage[utf8]{inputenc}
\usepackage[T1]{fontenc}
\usepackage[table]{xcolor}

\def\thesubsubsectiondis{\unskip\arabic{subsubsection})}

\usepackage{hyperref}
\hypersetup{
    colorlinks=true,
    linkcolor=blue,
    urlcolor=brown,
    }

\usepackage[nopostdot,nonumberlist,automake,symbols]{glossaries-extra}

\usepackage{ifthen}

\setabbreviationstyle{long-short}
\setglossarystyle{altlist}

\makeglossaries

% % only show link on first use
\GlsXtrEnableEntryUnitCounting{general}{0}{chapter}
\renewcommand*{\glslinkcheckfirsthyperhook}{%
    \ifnum\glsentrycurrcount\glslabel>0
        \setkeys{glslink}{hyper=false}%
    \fi
}

% #1 - short name: OIL
% #2 - math symbol
% #3 - full name: Open Information Layer
% #4 - description

\newcommand{\newdualentry}[4]{%
    \ifthenelse{\equal{#2}{}}%
    {
        % If #2 (the math symbol) is empty
        \newglossaryentry{#1}{%
            name={#3 (#1)},
            text={#1},
            short={#1},
            long={#3},
            first={#3 (#1)},
            firstplural={#3s (#1s)},
            description={#4}
        }
    }{
        % If #2 is provided
        \newglossaryentry{#1}{%
            name={#3 (\ensuremath{#2})},
            text={#1},
            short={\ensuremath{#2}},
            long={#3},
            first={#3 (\ensuremath{#2})},
            firstplural={#3s (#1s)},
            description={#4},
            symbol={\ensuremath{#2}}
        }
    }
}


\newdualentry{OW}{}{Open Web}{The next-generation Internet where information flows openly without any restrictions, as it is supposed to be.}

\newdualentry{OI}{}{Open Information}{Information that is typically found across various types of networks, including decentralized, federated, and centralized networks that allow permissionless access.}

\newdualentry{OIL}{}{Open Information Layer}{A conceptual layer where information flows openly without any restrictions.}

\newdualentry{DSL}{}{Data Sub-layer}{A decentralized network where the Open Information flows from its source to its destination.}

\newdualentry{UMS}{}{Unified Metadata Schemas}{A unified set of data structures for interoperability.}

\newdualentry{SN}{}{Serving Node}{A Data Sub-layer component that indexes, cleans, stores, and ultimately serves the Open Information to the end users.}

\newdualentry{GI}{}{Global Indexer}{A Data Sub-layer component that facilitates coordination among Serving Nodes and engages with the Value Sub-layer.}

\newdualentry{VSL}{}{Value Sub-layer}{A blockchain where the value created by Open Information activities is recorded and distributed.}

\newdualentry{PDS}{}{Permissionless Data Source}{A repository of data that can be accessed without the need for authorization or authentication.}

\newdualentry{NO}{}{Node Operator}{An individual or organization that operates a Serving Node.}

\newdualentry{PGP}{P_p}{Public Good Pool}{A collective pool of staked \$RSS3 that is used to improve the RSS3 Network by assigning trust to Public Good Nodes.}

\newdualentry{OP}{P_o}{Operating Pool}{A pool of \$RSS3 that consists of 1) Fees collected from serving Data Sub-layer requests; 2) Network Rewards allocated based on the Node's work.}

\newdualentry{SP}{P_s}{Staking Pool}{A pool of staked \$RSS3 that is used to improve the RSS3 Network by assigning trust to Normal Nodes.}

\newdualentry{R3N}{}{RSS3 Network}{A decentralized network that is formed by a DSL and a VSL.}

\begin{document}

\title{RSS3: The Open Information Layer}

\author{Natural Selection Labs}
\maketitle

\thispagestyle{plain}
\pagestyle{plain}
\pagenumbering{arabic}

\begin{abstract}

Inspired by the original RSS Standard, this paper presents RSS3, \acrlong{OIL} for the \acrlong{OW}. The paper serves as an enhanced version of our initial whitepaper titled ``RSS3: A Next-Generation Feed Standard.'' Following the release of our initial whitepaper, we have adhered to its proposed architecture to conduct experiments and advance the development of the RSS3 Network. The Network has transformed into what is now known as the \acrlong{OIL}, reflecting the evolving dynamics of the \acrlong{OW}. This paper summarizes our research and development progress since then, providing insights into RSS3's vision and its decentralization architecture.

\end{abstract}

\section{Introduction} 

RSS3 is the \glsfmtlong{OIL}, structuring \glsfmtlong{OI} for social, search, and AI. The \gls{OIL} is a conceptual layer where information flows openly without any restrictions, as it is supposed to be.

It is RSS3's mission to construct the \glsfmtlong{OW} by enhancing the free flow of \glsfmtlong{OI}.



\section{Open Information Layer}

The \gls{OIL} is a conceptual layer that is formed by two sub-layers: the \glsfirst{DSL} and the \glsfirst{VSL}. Information from permissonless data sources on the \gls{OIL} flows openly without any restrictions.


\section{Data Sub-layer}
\label{sec:DSL}

The \gls{DSL} is responsible for information life cycle management, which includes indexing, transformation, storage, dissemination, and consumption \cite{nationalinstituteofstandardsandtechnology2016Information}. The \gls{DSL} is formed by two components (see \cref{subsec:SN} and \cref{subsec:GI}), and uses the \gls{UMS} (see \cref{subsec:UMS}) to structure the information.

\subsection{\glsfirst{SN}}
\label{subsec:SN}

An \gls{SN} is responsible for indexing, cleaning, storing, and ultimately serving the Open Information to the end users. Each \gls{SN} operates a number of workers that index and structure information from \gls{PDS}, stores the information, and provides interfaces for access.

\subsection{\glsfirst{GI}}
\label{subsec:GI}

An \gls{GI} is responsible for facilitating coordination among \glspl{SN} and engaging with the \gls{VSL}, and performs the following functions:
\begin{enumerate}
    \item A load balancer and query router for end users to retrieve information from \glspl{SN}.
    \item A supervisor for \glspl{SN} to ensure the quality of service.
    \item A settler for submitting work and slash records to the \gls{VSL}.
\end{enumerate}


\subsection{\glsfirst{UMS}}
\label{subsec:UMS}

Open Information, indexed from multiple \glspl{PDS}, is structured by \glspl{SN} into the \gls{UMS} format for interoperability.

\glspl{PDS} use different data structures, within a \gls{PDS}, there might be multiple products, services and protocols that leverage a different data structure to suit their needs. This means limited interoperability, and developers need to look into each and every data structure, when it comes to building. This lack of standardization means developers must investigate each unique structure individually when building applications, which is not scalable.

The \gls{UMS} addresses this issue by offering a unified set of data structures that serve as an abstraction. This abstraction simplifies the integration process, making it more manageable and scalable for developers to work with data across various data sources.

For the complete set of the \gls{UMS}, refer to \url{https://docs.rss3.io/docs/unified-metadata-schemas}.



\section{\glsfmtlong{VSL}}
\label{sec:VSL}

The \glsfirst{VSL} is an Ethereum Layer 2 blockchain built with OP Stack uisng Celestia as the data availability layer. It is responsible for handling value derived from Open Information applications, such as social, search, AI and beyond.


\section{Incentive}

\subsection{Incentivization}

The RSS3 Network, on the other hand, will be rewarding network participants with the profit of the network generated from advertising, value-added services, social economic activities, etc.

\subsection{Staking and Slashing}

\subsection{Incentive Pool}

\subsubsection{Operator Pool}

\subsubsection{Reward Pool}


\section{Scalability}

As a network grows, the performance bottleneck often results in a slow processing speed and a higher transaction cost. In previous sections, subgroup scaling (Sec.~\ref{subsubsec:{Subgroup Scaling}}) and SDG (Sec.~\ref{subsubsec:{Scalable Dynamic Grouping (SDG)}}) are described as measures to maintain network's availability and usability, they are also designed to improve both storage and communication efficiency.

\subsection{Storage Efficiency}

Storage efficiency is critical for a decentralized network. First of all, RSS3 Files are lightweight by design (since they contain the metadata only). DAO limits the number of RSS3 Files hosted by an SN and dynamically increases the number of subgroups through subgroup scaling. This strategy provides a sufficient level of data redundancy while maintaining storage efficiency.

\subsection{Communication Efficiency}

In any decentralized network, an increase in efficiency leads to a reduction in fault tolerance. Extensive research has been done to ensure that the RSS3 Network is able to maintain an ideal equilibrium. As opposed to a blockchain based network\cite{bitcoin-whitepaper, eth-whitepaper}, where a global consensus is required, an RSS3 Network only requires each subgroup to reach an internal consensus. This significantly reduces the communication complexity. As the number of RSS3 Files increases, DAO dynamically scales the network through the election mechanism and subgroup scaling in conjunction with SDG performed by GIs, to further reduce communication complexity for subgroups.

GIs need to reach an internal consensus from time to time - such a consensus only requires an aggregated signature to reduce communication complexity. Furthermore, state-of-the-art BFT algorithms are implemented to maximize communication efficiency\cite{Mir-BFT,ibft}.

More subgroups will inevitably overwhelm GIs. On top of scaling through the election mechanism, this is additionally mitigated by increasing the number of RNs to take the workload of client request handling (where no consensus is needed) off GIs' shoulders. DAO further imposes a limit on the maximum number of GIs to improve communication efficiency.


\section{Conclusion} 

\textbf{At the heart of Natural Selection Labs, we firmly believe in the freedom of information allocation: No organizations or authorities shall prohibit the free exercise of the right of people to create, store, and distribute their information.}


\printglossary[type=main,title=Glossary, toctitle=Glossary]



% \bibliographystyle{plain}
% \bibliography{references}

\end{document}

